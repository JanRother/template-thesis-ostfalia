% PAGE LAYOUT
% requires packages from "packages.tex"



\selectlanguage{ngerman}								% set language to german

\setlength{\parindent}{0em} 							% paragraph indentation to left-justified

\onehalfspacing											% set line spacing to 1.5

\makeatletter

% configure headers
\def\maxChapterTitleLength{44}							% define maximum length of chapter title in header

\patchcmd{\@makechapterhead}{\vspace*{50\p@}}{}{}{} 	% removes space above \chapter head
\patchcmd{\@makeschapterhead}{\vspace*{50\p@}}{}{}{} 	% removes space above \chapter* head
\makeatother
\setlength{\headheight}{14.5pt}							% set header height
\pagestyle{fancy}										% set page style to fancy

\renewcommand{\chaptermark}[1]{
	\noexpandarg
	\StrLen{#1}[\chapterTitleLength]

	\ifthenelse{\chapterTitleLength > \maxChapterTitleLength}{
		\StrLeft{#1}{\maxChapterTitleLength}[\title]
        \edef\title{\title \ldots}
	}{
		\def\title{#1}
	}

	\markboth{\title\ -- \chaptername\ \thechapter\ }{}	% set chapter mark
}

\renewcommand{\sectionmark}[1]{							% set section mark
	\markright{
		{\projectShortDescription}
	}
}

\fancypagestyle{titlepage}								% set title page style
{
	\setcounter{page}{-100000}
	\fancyhf{}
	\fancyfootoffset{0pt}
	\fancyheadoffset{0pt}
	\renewcommand{\headrulewidth}{0pt}
}

% set fonts of titles
\titleformat{\chapter}[display] {\sffamily \huge}{\chaptertitlename\ \thechapter}{-5pt}{\Huge}
\titlespacing*{\chapter}{0pt}{0pt}{10pt}
\titleformat{\section}[display] {\sffamily \tiny}{}{0pt}{\LARGE \thesection\ }
\titlespacing*{\section}{0pt}{0pt}{0pt}
\titleformat{\subsection}[display] {\sffamily \tiny}{}{-15pt}{\Large \thesubsection\ }
\titlespacing*{\subsection}{0pt}{0pt}{0pt}
\titleformat{\subsubsection}[display] {\sffamily \tiny}{}{-15pt}{\large \thesubsubsection\ }
\titlespacing*{\subsubsection}{0pt}{0pt}{0pt}

% set fonts of table of contents
\renewcommand{\cftchapfont}{\bf\sffamily}
\renewcommand{\cftsecfont}{\sffamily}
\renewcommand{\cftsubsecfont}{\sffamily}

% configure glossary and acronyms
% -> styles: https://www.dickimaw-books.com/gallery/glossaries-styles/
\setglossarystyle{altlistgroup}
\setacronymstyle{long-short}

% set fonts of footnotes
\renewcommand{\cfttabfont}{\sffamily}
\renewcommand{\cftfigfont}{\sffamily}

\setcounter{secnumdepth}{3}

% set metadata of document
\hypersetup{
	pdfauthor={\firstAuthor},
	pdftitle={\documentTitle\ -- \documentSubtitle},
	pdfsubject={\documentSubject \documentYear},
	pdfkeywords={Betreut von \tutorAName und \tutorBName.}
}

% set code listings
\lstset{
	showspaces=false,
	showtabs=false,
	breaklines=true,
	showstringspaces=false,
	basicstyle=\ttfamily,
	frame=lt,
	rulecolor=\color{gray},
	framerule=3pt,
	xleftmargin=6pt,
}

% set code boxes
\newtcblisting{codebox}[3][]{%
	listing engine=minted,
	minted style=colorful,
	minted language=#2,
	minted options={tabsize=2,breaklines,autogobble,linenos,numbersep=3mm},
	colback=white,
	colframe=darkgray,
	listing only,
	left=5mm,
	title=\faCode\quad #3,
	enhanced,
	overlay={\begin{tcbclipinterior}\fill[lightgray] (frame.south west)
	rectangle ([xshift=5mm]frame.north west);\end{tcbclipinterior}},
	#1
}

% style captions
\newlength\myx
\setlength\myx{\textwidth}
\addtolength\myx{-2\fboxsep}
\DeclareCaptionFont{white}{\color{white} \sffamily}
\DeclareCaptionFormat{listing}{\colorbox{gray}{\parbox{\myx}{#1#2#3}}}
\captionsetup[lstlisting]{format=listing,labelfont=white,textfont=white}
\renewcommand{\lstlistingname}{Quelltext}